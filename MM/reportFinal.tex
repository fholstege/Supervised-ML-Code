% Options for packages loaded elsewhere
\PassOptionsToPackage{unicode}{hyperref}
\PassOptionsToPackage{hyphens}{url}
%
\documentclass[
]{article}
\usepackage{lmodern}
\usepackage{amssymb,amsmath}
\usepackage{ifxetex,ifluatex}
\ifnum 0\ifxetex 1\fi\ifluatex 1\fi=0 % if pdftex
  \usepackage[T1]{fontenc}
  \usepackage[utf8]{inputenc}
  \usepackage{textcomp} % provide euro and other symbols
\else % if luatex or xetex
  \usepackage{unicode-math}
  \defaultfontfeatures{Scale=MatchLowercase}
  \defaultfontfeatures[\rmfamily]{Ligatures=TeX,Scale=1}
\fi
% Use upquote if available, for straight quotes in verbatim environments
\IfFileExists{upquote.sty}{\usepackage{upquote}}{}
\IfFileExists{microtype.sty}{% use microtype if available
  \usepackage[]{microtype}
  \UseMicrotypeSet[protrusion]{basicmath} % disable protrusion for tt fonts
}{}
\makeatletter
\@ifundefined{KOMAClassName}{% if non-KOMA class
  \IfFileExists{parskip.sty}{%
    \usepackage{parskip}
  }{% else
    \setlength{\parindent}{0pt}
    \setlength{\parskip}{6pt plus 2pt minus 1pt}}
}{% if KOMA class
  \KOMAoptions{parskip=half}}
\makeatother
\usepackage{xcolor}
\IfFileExists{xurl.sty}{\usepackage{xurl}}{} % add URL line breaks if available
\IfFileExists{bookmark.sty}{\usepackage{bookmark}}{\usepackage{hyperref}}
\hypersetup{
  pdftitle={Which Variables Contribute to Air Quality? Evidence From California},
  hidelinks,
  pdfcreator={LaTeX via pandoc}}
\urlstyle{same} % disable monospaced font for URLs
\usepackage[margin=1in]{geometry}
\usepackage{color}
\usepackage{fancyvrb}
\newcommand{\VerbBar}{|}
\newcommand{\VERB}{\Verb[commandchars=\\\{\}]}
\DefineVerbatimEnvironment{Highlighting}{Verbatim}{commandchars=\\\{\}}
% Add ',fontsize=\small' for more characters per line
\usepackage{framed}
\definecolor{shadecolor}{RGB}{248,248,248}
\newenvironment{Shaded}{\begin{snugshade}}{\end{snugshade}}
\newcommand{\AlertTok}[1]{\textcolor[rgb]{0.94,0.16,0.16}{#1}}
\newcommand{\AnnotationTok}[1]{\textcolor[rgb]{0.56,0.35,0.01}{\textbf{\textit{#1}}}}
\newcommand{\AttributeTok}[1]{\textcolor[rgb]{0.77,0.63,0.00}{#1}}
\newcommand{\BaseNTok}[1]{\textcolor[rgb]{0.00,0.00,0.81}{#1}}
\newcommand{\BuiltInTok}[1]{#1}
\newcommand{\CharTok}[1]{\textcolor[rgb]{0.31,0.60,0.02}{#1}}
\newcommand{\CommentTok}[1]{\textcolor[rgb]{0.56,0.35,0.01}{\textit{#1}}}
\newcommand{\CommentVarTok}[1]{\textcolor[rgb]{0.56,0.35,0.01}{\textbf{\textit{#1}}}}
\newcommand{\ConstantTok}[1]{\textcolor[rgb]{0.00,0.00,0.00}{#1}}
\newcommand{\ControlFlowTok}[1]{\textcolor[rgb]{0.13,0.29,0.53}{\textbf{#1}}}
\newcommand{\DataTypeTok}[1]{\textcolor[rgb]{0.13,0.29,0.53}{#1}}
\newcommand{\DecValTok}[1]{\textcolor[rgb]{0.00,0.00,0.81}{#1}}
\newcommand{\DocumentationTok}[1]{\textcolor[rgb]{0.56,0.35,0.01}{\textbf{\textit{#1}}}}
\newcommand{\ErrorTok}[1]{\textcolor[rgb]{0.64,0.00,0.00}{\textbf{#1}}}
\newcommand{\ExtensionTok}[1]{#1}
\newcommand{\FloatTok}[1]{\textcolor[rgb]{0.00,0.00,0.81}{#1}}
\newcommand{\FunctionTok}[1]{\textcolor[rgb]{0.00,0.00,0.00}{#1}}
\newcommand{\ImportTok}[1]{#1}
\newcommand{\InformationTok}[1]{\textcolor[rgb]{0.56,0.35,0.01}{\textbf{\textit{#1}}}}
\newcommand{\KeywordTok}[1]{\textcolor[rgb]{0.13,0.29,0.53}{\textbf{#1}}}
\newcommand{\NormalTok}[1]{#1}
\newcommand{\OperatorTok}[1]{\textcolor[rgb]{0.81,0.36,0.00}{\textbf{#1}}}
\newcommand{\OtherTok}[1]{\textcolor[rgb]{0.56,0.35,0.01}{#1}}
\newcommand{\PreprocessorTok}[1]{\textcolor[rgb]{0.56,0.35,0.01}{\textit{#1}}}
\newcommand{\RegionMarkerTok}[1]{#1}
\newcommand{\SpecialCharTok}[1]{\textcolor[rgb]{0.00,0.00,0.00}{#1}}
\newcommand{\SpecialStringTok}[1]{\textcolor[rgb]{0.31,0.60,0.02}{#1}}
\newcommand{\StringTok}[1]{\textcolor[rgb]{0.31,0.60,0.02}{#1}}
\newcommand{\VariableTok}[1]{\textcolor[rgb]{0.00,0.00,0.00}{#1}}
\newcommand{\VerbatimStringTok}[1]{\textcolor[rgb]{0.31,0.60,0.02}{#1}}
\newcommand{\WarningTok}[1]{\textcolor[rgb]{0.56,0.35,0.01}{\textbf{\textit{#1}}}}
\usepackage{graphicx,grffile}
\makeatletter
\def\maxwidth{\ifdim\Gin@nat@width>\linewidth\linewidth\else\Gin@nat@width\fi}
\def\maxheight{\ifdim\Gin@nat@height>\textheight\textheight\else\Gin@nat@height\fi}
\makeatother
% Scale images if necessary, so that they will not overflow the page
% margins by default, and it is still possible to overwrite the defaults
% using explicit options in \includegraphics[width, height, ...]{}
\setkeys{Gin}{width=\maxwidth,height=\maxheight,keepaspectratio}
% Set default figure placement to htbp
\makeatletter
\def\fps@figure{htbp}
\makeatother
\setlength{\emergencystretch}{3em} % prevent overfull lines
\providecommand{\tightlist}{%
  \setlength{\itemsep}{0pt}\setlength{\parskip}{0pt}}
\setcounter{secnumdepth}{-\maxdimen} % remove section numbering

\title{Which Variables Contribute to Air Quality? Evidence From California}
\author{}
\date{\vspace{-2.5em}}

\begin{document}
\maketitle

\hypertarget{introduction}{%
\subsection{Introduction}\label{introduction}}

The World Health Organization (WHO) estimates 7 million that air
pollution contributes to the death of 7 million people, anually (WHO
2020). For policymakers, it is thus impportant to understand which
factors contribute to air quality, in order to target their efforts to
improve it. This study aims to pinpoint the factors that contribute to
air quality, using multiple regression. The factors considered in this
study are rainfall, population density, income per capita, added value
of companies and adjacency to coast. To find which combination of these
variables best described their relationship on air quality, we use the
better subset selection algorithm (Xiong 2014). Previous work showed
this algorithm yields a better fit than a subset without optimization as
the result of its monotonicity (Xiong 2014). Our data revealed that
whether or not an area is coastal was the only statistically significant
variable.

\hypertarget{Data}{%
\subsection{Data}\label{Data}}

Previous work has suggested the relation between the five chosen
variables and air quality. Both natural and anthropogenic events
attribute to air quality in the atmosphere. The distribution of air
pollution mainly depends on the wind field (Leelőssy et al. 2014), which
is quantified by the variable of adjacency to coast and rainfall in this
study as they both reflect the wind field's condition. Air quality is
also influenced by the production and consumption from society, leading
to emmissions (Baklanov, Molina, and Gauss 2016). To account for this,
our study includes variables on population density, income per capita
and added values from companies. We use the econometrics dataset on air
quality in California for 1972 (r-project 2020). The dependent variable
is an indicator of air quality (Ruoying - can you find what this
indicator consists of?), the lower the better. The independent variables
under study are rainfall(inch), population density(per square mile),
income per capita (\$), added value of companies (\$) and adjacency to
coast(binary). The dataset has 30 set of observations, each being a
different metropolitan area in California. As the unit of each variable
is heterogenerous, we scaled all independent variables to have a mean of
0 and a variance of 1, to ensure fair interpretation of the model
coefficents. Scaling also ensures the MM algorithm converges faster.
\textbackslash{}

Looking at the correlations between the variables, the only one that
stands out is median income and value added per company with a
correlation of 0.89 (see table below). This means we need to be careful
with models which include both of these variables, since
multicollinearity might lead to a wrong interpretation of the
coefficients.

\begin{table}[!htbp] \centering 
\small
  \label{} 
\begin{tabular}{@{\extracolsep{5pt}} cccccc} 
\\[-1.8ex]\hline 
\hline \\[-1.8ex] 
 & Value Added & Rain & Coastal Area & Population Density & Median Income \\ 
\hline \\[-1.8ex] 
Value Added & 1 & $-$0.149 & 0.010 & 0.158 & 0.890 \\ 
Rain &  & 1 & 0.185 & 0.009 & -0.086 \\ 
Coastal Area & &  & 1 & 0.005 & 0.170 \\ 
Population Density &  &  &  & 1 & 0.195 \\ 
Median Income &  &  &  &  & 1 \\ 
\hline \\[-1.8ex] 
\end{tabular} 
\end{table}

\hypertarget{Method}{%
\subsection{Method}\label{Method}}

Multiple regression makes a linear combination of several explanatory
predictors to predict the outcome of a response variable Y:

\begin{align}
Y = \beta_0 + \beta_1 &X_1+\beta_2 X_2 +...+ \beta_p X_p + \epsilon,
\\
\boldsymbol{y} = \mathbf{X} \boldsymbol{\beta} + \epsilon,
\end{align}

In (1), Y denotes a random response variable, X denotes random vectors
of p predictor variables. This can also be written in matrix form in
(2), where \(\boldsymbol{y}\) denotes an nX1 column vector,
\(\mathbf{X}\)an include n\(\times\)(p+1) matrix of thepredictor
variables with a first column of 1 for the intercept, and
\(\boldsymbol{\beta}\) denotes (p+1) vector of
weights\([\beta_0, \beta_1, \beta_2,...,\beta_p]^\top\). In order to
find the set of \(\boldsymbol{\beta}\) that best fits the model, we need
to solve the following minimization problem:

\begin{align}
 RSS (\boldsymbol{\beta}) = \mathbf{e}^\top \mathbf{e} = (\mathbf{y-X\boldsymbol{\beta}})&^\top (\mathbf{y-X\boldsymbol{\beta})},
\\
=\mathbf{y}^\top \mathbf{y} + \boldsymbol{\beta}^\top\mathbf{X}^\top\mathbf{X}\boldsymbol{\beta} -2\boldsymbol{\beta}\mathbf{X}^\top \mathbf{y}.
\end{align}

The difficult part of minimizing \(RSS (\boldsymbol{\beta})\) lies in
\(\boldsymbol{\beta}^\top\mathbf{X}^\top\mathbf{X}\boldsymbol{\beta}\),
since computing \(\mathbf{X}^\top\mathbf{X}\) is computationally
expensive with a large sample. One workaround for this problem is to
minimize \(RSS (\boldsymbol{\beta})\) using the majorization in
minimization algorithm (MM). For this algorithm, we need to find a
majorizing function of the form
\(\lambda\boldsymbol{\beta}^\top\boldsymbol{\beta} - 2\boldsymbol{\beta}^\top\beta\).
For this majorizing function, we need
\(\mathbf{X}^\top\mathbf{X}-\lambda\mathbf{I}\) to be negative
semidefinite (nsd). If this condition is unmet, the majorizing function
is not convex, and thus we cannot find a minimum. We know that
\(\mathbf{X}^\top\mathbf{X}-\lambda\mathbf{I}\) is nsd for
\(\lambda\geq\lambda_{max}\) with \(\lambda_{max}\) is the largest
eigenvalue of \(\mathbf{X}^\top\mathbf{X}\). \textbackslash{}

If we substitute the majorizing function into \$RSS (\boldsymbol{\beta})
\$, and minimize this new function (Ruoying adds notation here)

We try to find the \(\boldsymbol{\beta}\) for which RSS is minimized. We
do this by choosing with an random initial \(\beta_0\in\mathbb{R}^p\).
We improve \(\boldsymbol{\beta}\) through iteration k iterations. We
define the improvement at step k is defined as the step score:

\textbackslash begin\{aligh\}

\text{step score} =
\$\frac{RSS (\boldsymbol{\beta_{k-1}}) - RSS (\boldsymbol{\beta_{k}})}{RSS (\boldsymbol{\beta_{k-1}})}
\$

\textbackslash end\{align\}

If the step score falls below \(\epsilon\), we stop the iteration to
minimize \$RSS (\boldsymbol{\beta}) \$. We set the \(\epsilon\) to a
very small number, to ensure that the function is minimized until only
very minor improvements can be made. This brings our final solution
closer to the optimal solution. \textbackslash{}

To judge which model best explains the relationship between the
independent and dependent variables, we examine the adjusted R\^{}2. To
explain why we use this statistic, let's first examine \(R^2\), which iS
defined as \(R^2 = 1 - \frac{SSE}{SSTO}\), with SSTO and SSE denoting
total sum of squares and error sum of squares correspondingly. This is a
measure of the \% of variance in the dependent variable that can be
explained by the model. When more variables are added, \(R^2\)
inherently will increase, but this might due to random chance. To
account for this, we look at adjusted \(R^2\), defined as
\(R^2_\alpha = 1 - (\frac{n-1}{n-p})(\frac{SSE}{SSTO})\), with n
denoting our sample size, and p the number of independent variables
minus the intercept. \textbackslash{}

To find which models have a high \(R^2_\alpha\), we use the better
subset algorithm as proposed by (Xiong 2014). In this algorithm, we
again use MM to minimize \$RSS (\boldsymbol{\beta}) \$, but only allow
models with a number of \(m\) variables, as a subset of the original
indepdent variables. This goes as follows: we set \(m\) initial random
values, with \(\beta_0\in\mathbb{R}^p\). Once we calculated \(\beta_k\),
we sort \(|\beta_k|\), and only leave the \(m\) largest coefficients as
non-zero. We then continue iterating on \(|\beta_k|\) with this new set
of coefficients. In our case, we let \(m \in [1,5]\), as there were 5
independent variables in our dataset. We then computed for each \(m\)
the best model according to better subset selection.

\hypertarget{Result}{%
\subsection{Result}\label{Result}}

The model that best describes the relationship (e.g.~with the highest
adjusted \(R^2\)) is the one with just coastal area, and value added per
company. This model does not suffer from multicollinearity, since the
value added variable and the coastal area variable are not correlated.
Across all the models, the only variable that is statistically
significant is the coastal area variable. The VIF of this last model is
also below 5 (\ldots), which according to \ldots{} means there is no
multicollinearity. The table below shows the results of the standardized
independent variables on the dependent variable of Y quality. The
coefficient in this case tells us that one standard deviation of change
in the independent variable, leads to a certain change in the dependent
variable.

\begin{table}[!htbp] \centering 
  \label{} 
\begin{tabular}{@{\extracolsep{5pt}}lccccc} 
\\[-1.8ex]\hline 
\hline \\[-1.8ex] 
 & \multicolumn{5}{c}{\textit{Dependent variable:}} \\ 
\cline{2-6} 
\\[-1.8ex] & \multicolumn{5}{c}{Air Quality} \\ 
\\[-1.8ex] & (1) & (2) & (3) & (4) & (5)\\ 
\hline \\[-1.8ex] 
 Coastal Area (1 = coastal, 0 = non-coastal) & $-$13.73$^{***}$ & $-$13.83$^{***}$ & $-$16$^{***}$ & $-$14.67$^{***}$ & $-$15.54$^{***}$ \\ 
  & (-2.54) & ($-$2.93) & ($-$3.37) & ($-$3.00) & ($-$3.19) \\ 
  & & & & & \\ 
 Value Added & - & 9.26 & - & 10.19 & 4.24 \\ 
  &  & (0.92) &  & (0.98) & (0.41) \\ 
  & & & & & \\ 
 Median Income & - & - & 10.0 & - & 6.9 \\ 
  &  &  & (0.969) &  & (0.64) \\ 
  & & & & & \\ 
 Population Density & - & - & - & $-$2.66 & $-$3.04 \\ 
  &  &  &  & ($-$0.58) & ($-$0.66) \\ 
  & & & & & \\ 
  Rain  & - & - & - & - & 3.38 \\ 
  &  &  &  & & 0.73 \\ 
  & & & & & \\ 
 Intercept & 104.700$^{***}$ & 104.700$^{***}$ & 104.700$^{***}$ & 104.700$^{***}$ & 104.700$^{***}$ \\ 
  & (21.34) & (23.075) & (23.24) & (23.43) & (23.69) \\ 
  & & & & & \\ 
\hline \\[-1.8ex] 
Observations & 30 & 30 & 30 & 30 & 30 \\ 
R$^{2}$ & 0.240 & 0.349 & 0.359 & 0.369 & 0.383 \\ 
Adjusted R$^{2}$ & 0.21 & 0.30 & 0.29 & 0.27 & 0.25 \\ 
\hline 
\hline \\[-1.8ex] 
\textit{Note:}  & \multicolumn{5}{r}{$^{*}$p$<$0.1; $^{**}$p$<$0.05; $^{***}$p$<$0.01} \\ 
\end{tabular} 
\end{table}

To make the result on coastal areas a bit more interpretable, If we
rescale the coefficient of the coastal area (by adding the mean, and
dividing by the standard deviation), we can infer that being an coastal
area leads to change of-28.17 points in the air quality index, which
means that coastal areas have higher air quality.

\hypertarget{limitations-and-future-research}{%
\subsection{Limitations and Future
Research}\label{limitations-and-future-research}}

Apart from the limitations to our dataset, we should note several other
limitations to our applied method. First, there are other variables that
affect air polution that we did not consider. An example is \ldots..
(add citation). Second, while our method is appropriate for estimating
linear relationships, previous research has shown that the effects of
certain variables on air pollution is non-linear - one such example is
environmental regulation (Liu, Luo, and Wu 2019). \textbackslash{}
\textbackslash{} Future research should aim to add more areas, years,
and variables to our analysis to consider other possibilities. In
addition, while there is some existing research on air quality in
coastal areas (see\ldots.), it is worth further disentangling the
geographic aspects of coastal areas that lead to an improvement in air
quality, to see if those aspects that can be recreated in non-coastal
areas.

\hypertarget{conclusion}{%
\subsection{Conclusion}\label{conclusion}}

(add sentence) Our results suggest that coastal areas experience much
better air quality than non-coastal areas. This has implications for
policymakers. It provides evidence for initiatives that move residents
to coastal areas to expose them to enhanced air quality. One such
initiative was recently announced in Taiwan(\ldots..). On the other
hand, our results suggest that policies which aim to limit population
density or economic activity might be less effective than previously
thought, since we found no statistically significant relationship
between these variables and air pollution. \textbackslash{}

\hypertarget{Reference}{%
\subsection{Reference}\label{Reference}}

\hypertarget{functions}{%
\subsubsection{Functions}\label{functions}}

\begin{Shaded}
\begin{Highlighting}[]
\CommentTok{# calcRSS: Calculates the residual squared errors for a multiple regression of the form Y = XBeta + e}
\CommentTok{# }
\CommentTok{# Parameters: }
\CommentTok{#   mX: Matrix of n x p (n = observations, p = independent variables)}
\CommentTok{#   mY: Column matrix of n x 1 dependent variables (n = observations)}
\CommentTok{#   mBeta: Column Matrix of p x 1 coefficients}
\CommentTok{#   }
\CommentTok{# Output:}
\CommentTok{#   ESquared: double, residual squared errors}

\NormalTok{calcRSS <-}\StringTok{ }\ControlFlowTok{function}\NormalTok{(mX, mY,mBeta)\{}
  
  \CommentTok{# calculate the errors}
\NormalTok{  mE <-}\StringTok{  }\NormalTok{mY }\OperatorTok{-}\StringTok{ }\NormalTok{mX }\OperatorTok\StringTok{ }\NormalTok{mBeta}
  
  \CommentTok{# get errors squared}
\NormalTok{  ESquared <-}\StringTok{ }\KeywordTok{t}\NormalTok{(mE) }\OperatorTok\StringTok{ }\NormalTok{mE}
  
  \CommentTok{# return the residual sum of squared errors}
  \KeywordTok{return}\NormalTok{(ESquared[}\DecValTok{1}\NormalTok{,}\DecValTok{1}\NormalTok{])}
  
\NormalTok{\}}

\CommentTok{# calcCovar: Calculates the covariance matrix }
\CommentTok{#}
\CommentTok{# Parameters: }
\CommentTok{#   RSS: Residual squared errors}
\CommentTok{#   mXtX: pxp matrix, created from independent variables (X), multiplied with itself}
\CommentTok{#   n: double, number of observations}
\CommentTok{#   p: double, number of variables}
\CommentTok{#}
\CommentTok{# Output:}
\CommentTok{#   Covar: matrix, covariance matrix}

\NormalTok{calcCovar <-}\StringTok{ }\ControlFlowTok{function}\NormalTok{(RSS, mXtX,n, p)\{}
  
  \CommentTok{# est. for sigma squared}
\NormalTok{  SigmaSquared <-}\StringTok{ }\NormalTok{(RSS) }\OperatorTok{/}\StringTok{ }\NormalTok{(n }\OperatorTok{-}\StringTok{ }\NormalTok{p }\DecValTok{-1}\NormalTok{)}
  
\NormalTok{  Covar <-}\StringTok{ }\NormalTok{SigmaSquared }\OperatorTok{*}\StringTok{ }\KeywordTok{as.matrix}\NormalTok{(}\KeywordTok{inv}\NormalTok{(mXtX))}
  
  \KeywordTok{return}\NormalTok{(Covar)}
  
\NormalTok{\}}

\CommentTok{# calcSignificance: Calculates the statistical significance of a set of beta's}
\CommentTok{#}
\CommentTok{# Parameters: }
\CommentTok{#   RSS: Residual squared errors}
\CommentTok{#   mXtX: pxp matrix, created from independent variables (X), multiplied with itself}
\CommentTok{#   n: double, number of observations}
\CommentTok{#   p: double, number of variables}
\CommentTok{#   mBetaEst: matrix of estimated Beta's}
\CommentTok{#}
\CommentTok{# Output:}
\CommentTok{#   dfSignificance: dataframe, containing the results on statistical signficance}

\NormalTok{calcSignificance <-}\StringTok{ }\ControlFlowTok{function}\NormalTok{(RSS, mXtX, n,p, mBetaEst)\{}
  
  \CommentTok{# get covariance matrix}
\NormalTok{  mCovar <-}\StringTok{ }\KeywordTok{calcCovar}\NormalTok{(RSS,mXtX,n,p)}
  
  \CommentTok{# calculate the standard deviations}
\NormalTok{  stdev <-}\StringTok{ }\KeywordTok{sqrt}\NormalTok{(}\KeywordTok{diag}\NormalTok{(mCovar))}
  
  \CommentTok{# define t, which is t-distributed with n-p-1 degrees of freedom }
\NormalTok{  t <-}\StringTok{ }\NormalTok{mBetaEst}\OperatorTok{/}\NormalTok{stdev}
\NormalTok{  pval <-}\StringTok{ }\DecValTok{2}\OperatorTok{*}\KeywordTok{pt}\NormalTok{(}\OperatorTok{-}\KeywordTok{abs}\NormalTok{(t),}\DataTypeTok{df=}\NormalTok{n}\OperatorTok{-}\NormalTok{p}\DecValTok{-1}\NormalTok{)}
  
\NormalTok{  dfSignificance <-}\StringTok{ }\KeywordTok{data.frame}\NormalTok{(}\DataTypeTok{BetaEst =}\NormalTok{ mBetaEst, }
                               \DataTypeTok{stdev =}\NormalTok{ stdev, }
                               \DataTypeTok{t =}\NormalTok{ t, }
                               \DataTypeTok{pval =}\NormalTok{ pval)}
  
  \KeywordTok{return}\NormalTok{(dfSignificance)}
\NormalTok{\}}

\CommentTok{# calcLargestEigen: Calculates the largest eigenvalue of an matrix of independent variables}
\CommentTok{# }
\CommentTok{# Parameters: }
\CommentTok{#   mX: Dataframe of n x p (n = observations, p = independent variables)}
\CommentTok{#   }
\CommentTok{# Output:}
\CommentTok{#   LargestEigenval: float, largest eigenvalue of said matrix}

\NormalTok{calcLargestEigen <-}\StringTok{ }\ControlFlowTok{function}\NormalTok{(mX)\{}
  
  \CommentTok{# get the eigenvalues of X }
\NormalTok{  EigenValX <-}\StringTok{ }\KeywordTok{eigen}\NormalTok{(mX)}\OperatorTok{$}\NormalTok{values}
  
  \CommentTok{# from these eigenvalues, get the largest one}
\NormalTok{  LargestEigenVal <-}\StringTok{ }\KeywordTok{max}\NormalTok{(EigenValX, }\DataTypeTok{na.rm =} \OtherTok{TRUE}\NormalTok{)}
  
  \KeywordTok{return}\NormalTok{(LargestEigenVal)}
  
\NormalTok{\}}

\CommentTok{# CalcStepScore: Calculates the % improvement between the k-1th and kth set of beta's}
\CommentTok{# }
\CommentTok{# Parameters:}
\CommentTok{#   prevBeta: double, k-1th beta}
\CommentTok{#   currbeta: double, kth beta}
\CommentTok{#   mX: Dataframe of n x p (n = observations, p = independent variables)}
\CommentTok{# }
\CommentTok{# Output: }
\CommentTok{#   StepScore; double, % improvement between the RSS of the two sets of beta's}

\NormalTok{calcStepScore <-}\StringTok{ }\ControlFlowTok{function}\NormalTok{(mX,mY, prevBeta, currBeta)\{}
  
  \CommentTok{# difference in RSS between previous and current set of beta's}
\NormalTok{  diffRSS <-}\StringTok{ }\NormalTok{(}\KeywordTok{calcRSS}\NormalTok{(mX,mY,prevBeta) }\OperatorTok{-}\StringTok{ }\KeywordTok{calcRSS}\NormalTok{(mX,mY,currBeta))}
  
  \CommentTok{# divide difference with previous score to get % change}
\NormalTok{  StepScore <-}\StringTok{ }\NormalTok{diffRSS }\OperatorTok{/}\KeywordTok{calcRSS}\NormalTok{(mX,mY,prevBeta)}
  
  \KeywordTok{return}\NormalTok{(StepScore)}
  
\NormalTok{\}}

\CommentTok{# calcRsquared}
\CommentTok{#}
\CommentTok{# Calculates the r-squared}
\CommentTok{#}
\CommentTok{# Parameters:}
\CommentTok{#   Y: matrix, the true dependent variable   }
\CommentTok{#   Yest: matrix, the predicted dependent variable}
\CommentTok{#   (optional) adjusted: if True, return adjusted r squared}
\CommentTok{#   (optional) p: if adjusted is calculated, add number of variables}
\CommentTok{# }
\CommentTok{# Output:}
\CommentTok{#   Rsquared: double, the Rsquared or adjusted Rsquared for a linear model}

\NormalTok{calcRsquared <-}\StringTok{ }\ControlFlowTok{function}\NormalTok{(mY, mYest, }\DataTypeTok{adjusted =} \OtherTok{FALSE}\NormalTok{, }\DataTypeTok{p=}\DecValTok{0}\NormalTok{, }\DataTypeTok{n=}\DecValTok{0}\NormalTok{)\{}
  
  \CommentTok{# standardize Y, and Yest (mean of 0)}
\NormalTok{  mStandY =}\StringTok{ }\NormalTok{mY }\OperatorTok{-}\StringTok{ }\KeywordTok{mean}\NormalTok{(mY)}
\NormalTok{  mStandYest =}\StringTok{ }\NormalTok{mYest }\OperatorTok{-}\StringTok{ }\KeywordTok{mean}\NormalTok{(mYest)}
  
  \CommentTok{# calculate Rsquared}
\NormalTok{  numerator <-}\StringTok{ }\NormalTok{(}\KeywordTok{t}\NormalTok{(mStandY) }\OperatorTok\StringTok{ }\NormalTok{mStandYest)}\OperatorTok{^}\DecValTok{2}
\NormalTok{  denominator <-}\StringTok{ }\NormalTok{(}\KeywordTok{t}\NormalTok{(mStandY) }\OperatorTok\StringTok{ }\NormalTok{mY) }\OperatorTok\StringTok{ }\NormalTok{(}\KeywordTok{t}\NormalTok{(mStandYest) }\OperatorTok\StringTok{ }\NormalTok{mStandYest)}
\NormalTok{  resultRsquared <-}\StringTok{ }\NormalTok{(numerator}\OperatorTok{/}\NormalTok{denominator)}
  
  \CommentTok{# if want adjusted R squared, }
  \ControlFlowTok{if}\NormalTok{(adjusted)\{}
    
\NormalTok{    adjRsquared =}\StringTok{ }\DecValTok{1} \OperatorTok{-}\StringTok{ }\NormalTok{(((}\DecValTok{1}\OperatorTok{-}\NormalTok{resultRsquared)}\OperatorTok{*}\NormalTok{(n }\OperatorTok{-}\StringTok{ }\DecValTok{1}\NormalTok{))}\OperatorTok{/}\NormalTok{(n}\OperatorTok{-}\NormalTok{p}\DecValTok{-1}\NormalTok{))}
\NormalTok{    resultRsquared <-}\StringTok{ }\NormalTok{adjRsquared}
\NormalTok{  \}}
  
  \KeywordTok{return}\NormalTok{(resultRsquared)}
  
\NormalTok{\}}

\CommentTok{# calcModelMM}
\CommentTok{#}
\CommentTok{# Calculates a linear model, using the majorization in minimization (MM) algorithm}
\CommentTok{#}
\CommentTok{# Parameters:}
\CommentTok{#   X: Dataframe of n x p (n = observations, p = independent variables)}
\CommentTok{#   Y: Dataframe of n x 1 dependent variables (n = observations)}
\CommentTok{#   e: epsilon, parameter for threshold of improvement after which the algorithm should halt}
\CommentTok{#   nBeta: number of variables one wants to use}
\CommentTok{#}
\CommentTok{# Output:}
\CommentTok{#   result: dataframe with attributes of the model: }
\CommentTok{#       - Beta: dataframe, the calculated Beta's}
\CommentTok{#       - RSS: double, Sum of squared residuals}
\CommentTok{#       - Yest: dataframe, the predicted Y}
\CommentTok{#       - Rsquared: double, R^2 for the predicted Y}
\CommentTok{#       - AdjRsquared: Adjusted Rsquared}
\CommentTok{#       - Significance results: dataframe with significance results on the beta's}
\CommentTok{#       - Residuals: dataframe, Y - Yest.}
\CommentTok{#}

\NormalTok{calcModelMM <-}\StringTok{ }\ControlFlowTok{function}\NormalTok{(mX,mY,e, nBeta)\{}
  
  \CommentTok{# get number of observations, and number of variables minues the intercept}
\NormalTok{  n <-}\StringTok{ }\KeywordTok{nrow}\NormalTok{(mX)}
\NormalTok{  p <-}\StringTok{ }\KeywordTok{ncol}\NormalTok{(mX) }\OperatorTok{-}\StringTok{ }\DecValTok{1}
  
  \CommentTok{# check the user has filled in an appropriate amount of beta's}
  \ControlFlowTok{if}\NormalTok{(nBeta }\OperatorTok{>}\StringTok{ }\NormalTok{p }\OperatorTok{+}\StringTok{ }\DecValTok{1}\NormalTok{)\{}
    \KeywordTok{stop}\NormalTok{(}\StringTok{"You want to use more variables than there are in the dataset of independent variables"}\NormalTok{)}
\NormalTok{  \}}
  
  \CommentTok{# set the previous beta variable to initial, random beta's}
\NormalTok{  prevBeta <-}\StringTok{ }\KeywordTok{runif}\NormalTok{(}\KeywordTok{ncol}\NormalTok{(mX), }\DataTypeTok{min=}\DecValTok{0}\NormalTok{, }\DataTypeTok{max=}\DecValTok{1}\NormalTok{)}
  
  \CommentTok{# calculate X'X}
\NormalTok{  mXtX <-}\StringTok{ }\KeywordTok{t}\NormalTok{(mX) }\OperatorTok\StringTok{ }\NormalTok{mX}
  
  \CommentTok{# get largest eigenvalue for the square of independent variables}
\NormalTok{  Lambda <-}\StringTok{ }\KeywordTok{calcLargestEigen}\NormalTok{(mXtX)}
  
  \CommentTok{# set initial stepscore to 0, k to 1. }
\NormalTok{  StepScore <-}\StringTok{ }\DecValTok{0}
\NormalTok{  k <-}\StringTok{ }\DecValTok{1}
  
  \CommentTok{# run while, either if k is equal to 1, or the improvement between k-1th and kth set of beta's is smaller than the parameter e}
  \ControlFlowTok{while}\NormalTok{ (k }\OperatorTok{==}\StringTok{ }\DecValTok{1} \OperatorTok{|}\StringTok{ }\NormalTok{StepScore }\OperatorTok{>}\StringTok{ }\NormalTok{e )\{}
    
    \CommentTok{# step to next k}
\NormalTok{    k <-}\StringTok{ }\NormalTok{k }\OperatorTok{+}\StringTok{ }\DecValTok{1}
    
    \CommentTok{# calculate beta's for this k}
\NormalTok{    BetaK <-}\StringTok{ }\NormalTok{prevBeta }\OperatorTok{-}\StringTok{ }\NormalTok{((}\DecValTok{1}\OperatorTok{/}\NormalTok{Lambda) }\OperatorTok{*}\StringTok{  }\NormalTok{mXtX }\OperatorTok\StringTok{ }\NormalTok{prevBeta) }\OperatorTok{+}\StringTok{ }\NormalTok{((}\DecValTok{1}\OperatorTok{/}\NormalTok{Lambda) }\OperatorTok{*}\StringTok{ }\KeywordTok{t}\NormalTok{(mX) }\OperatorTok\StringTok{ }\NormalTok{mY )}
    
    \CommentTok{# sort the beta's based on absolute value, remove the smallest ones to keep m }
\NormalTok{    absBetaKOrdered <-}\StringTok{ }\KeywordTok{order}\NormalTok{(}\KeywordTok{abs}\NormalTok{(BetaK[,}\DecValTok{1}\NormalTok{]), }\DataTypeTok{decreasing =}\NormalTok{ T)}
\NormalTok{    BetaK[}\OperatorTok{!}\NormalTok{BetaK }\OperatorTok\StringTok{ }\NormalTok{BetaK[absBetaKOrdered,][}\DecValTok{1}\OperatorTok{:}\NormalTok{nBeta]] <-}\StringTok{ }\DecValTok{0}
    
    \CommentTok{# new stepscore, % difference in RSS between new Beta's and previous beta's}
\NormalTok{    StepScore <-}\StringTok{ }\KeywordTok{calcStepScore}\NormalTok{(mX,mY,prevBeta,BetaK)}
    
    \CommentTok{# assign current beta's to prevBeta variable for next iteration}
\NormalTok{    prevBeta <-}\StringTok{ }\NormalTok{BetaK}
    
    
\NormalTok{  \}}
  
  \CommentTok{## Calculate several attributes of the linear model, put in dataframes or doubles}

  \CommentTok{# final Beta's}
\NormalTok{  BetaFinal <-}\StringTok{ }\KeywordTok{as.matrix}\NormalTok{(BetaK)}
  
  \CommentTok{# calculate the RSS of this final est.}
\NormalTok{  RSSBetaK <-}\StringTok{ }\KeywordTok{calcRSS}\NormalTok{(mX,mY, BetaK)}
  
  \CommentTok{# get the est. dependent variables}
\NormalTok{  mYest <-}\StringTok{ }\NormalTok{mX }\OperatorTok\StringTok{ }\NormalTok{BetaFinal}
  
  \CommentTok{# get the r2 and adjusted r2}
\NormalTok{  Rsquared <-}\StringTok{ }\KeywordTok{calcRsquared}\NormalTok{(mY, mYest)}
\NormalTok{  adjRsquared <-}\StringTok{ }\KeywordTok{calcRsquared}\NormalTok{(mY,mYest, }\DataTypeTok{adjusted =}\NormalTok{ T,  p, n)}
  
  \CommentTok{# get the residuals}
\NormalTok{  Resi <-}\StringTok{ }\NormalTok{mY }\OperatorTok{-}\StringTok{ }\NormalTok{mYest}
  
  \CommentTok{# get the results on significance}
\NormalTok{  dfSignificance <-}\StringTok{ }\KeywordTok{calcSignificance}\NormalTok{(RSSBetaK, mXtX, n, p, BetaFinal)}
  
  \CommentTok{# add these attributes together as a list to make it easily accessible}
\NormalTok{  result <-}\StringTok{ }\KeywordTok{list}\NormalTok{(}\DataTypeTok{Beta =}\NormalTok{ BetaFinal, }
                  \DataTypeTok{RSS =}\NormalTok{ RSSBetaK, }
                  \DataTypeTok{Yest =}\NormalTok{ mYest,}
                  \DataTypeTok{Rsquared =}\NormalTok{ Rsquared, }
                  \DataTypeTok{adjRsquared =}\NormalTok{ adjRsquared, }
                  \DataTypeTok{SignificanceResults =}\NormalTok{ dfSignificance,}
                  \DataTypeTok{Residuals =}\NormalTok{ Resi, }
                  \DataTypeTok{n =}\NormalTok{ n,}
                  \DataTypeTok{p =}\NormalTok{ p)}
  
  
  \KeywordTok{return}\NormalTok{(result)}
  
\NormalTok{\}}

\CommentTok{# findModelMM}
\CommentTok{#}
\CommentTok{# finds the best linear model, using the MM algorithm, by testing model with 1, 2...up to all variables in X}
\CommentTok{#}
\CommentTok{# Parameters:}
\CommentTok{#   mX: Matrix of n x p (n = observations, p = independent variables)}
\CommentTok{#   mY: Matrix of n x 1 dependent variables (n = observations)}
\CommentTok{#}
\CommentTok{# Output:}
\CommentTok{#   results: list with the results for each model version}

\NormalTok{findModelMM <-}\StringTok{ }\ControlFlowTok{function}\NormalTok{(mX, mY, e)\{}
  
  \CommentTok{# get the number of independent variables used}
\NormalTok{  nIndVar =}\StringTok{ }\KeywordTok{ncol}\NormalTok{(mX) }\OperatorTok{-}\StringTok{ }\DecValTok{1}
  
  \CommentTok{# start at m = 1, create empty list to be filled with results}
\NormalTok{  M =}\StringTok{ }\DecValTok{1}
\NormalTok{  results <-}\StringTok{ }\KeywordTok{list}\NormalTok{()}
  
  \CommentTok{# for each m, check the best model and save the results}
  \ControlFlowTok{while}\NormalTok{(M }\OperatorTok{<=}\StringTok{ }\NormalTok{nIndVar)\{}
    
\NormalTok{    M <-}\StringTok{ }\NormalTok{M }\OperatorTok{+}\StringTok{ }\DecValTok{1}
    
\NormalTok{    resultM <-}\StringTok{ }\KeywordTok{calcModelMM}\NormalTok{(mX, mY, e, M)}
    
\NormalTok{    strSave <-}\StringTok{ }\KeywordTok{paste0}\NormalTok{(}\StringTok{"Model with "}\NormalTok{, M}\DecValTok{-1}\NormalTok{, }\StringTok{" variable(s)"}\NormalTok{)}
\NormalTok{    results[[strSave]] <-}\StringTok{ }\NormalTok{resultM}
    
\NormalTok{  \}}
  
  \KeywordTok{return}\NormalTok{(results)}
  
\NormalTok{\}}
\end{Highlighting}
\end{Shaded}

\hypertarget{analysis}{%
\subsection{Analysis}\label{analysis}}

\begin{Shaded}
\begin{Highlighting}[]
\CommentTok{# load necessary packages}
\KeywordTok{library}\NormalTok{(matlib)}
\KeywordTok{library}\NormalTok{(stargazer)}
\end{Highlighting}
\end{Shaded}

\begin{verbatim}
## Warning: package 'stargazer' was built under R version 4.0.3
\end{verbatim}

\begin{verbatim}
## 
## Please cite as:
\end{verbatim}

\begin{verbatim}
##  Hlavac, Marek (2018). stargazer: Well-Formatted Regression and Summary Statistics Tables.
\end{verbatim}

\begin{verbatim}
##  R package version 5.2.2. https://CRAN.R-project.org/package=stargazer
\end{verbatim}

\begin{Shaded}
\begin{Highlighting}[]
\KeywordTok{library}\NormalTok{(sjPlot)}
\end{Highlighting}
\end{Shaded}

\begin{verbatim}
## Warning: package 'sjPlot' was built under R version 4.0.3
\end{verbatim}

\begin{verbatim}
## Registered S3 methods overwritten by 'lme4':
##   method                          from
##   cooks.distance.influence.merMod car 
##   influence.merMod                car 
##   dfbeta.influence.merMod         car 
##   dfbetas.influence.merMod        car
\end{verbatim}

\begin{verbatim}
## Learn more about sjPlot with 'browseVignettes("sjPlot")'.
\end{verbatim}

\begin{Shaded}
\begin{Highlighting}[]
\KeywordTok{library}\NormalTok{(multiColl)}
\end{Highlighting}
\end{Shaded}

\begin{verbatim}
## Warning: package 'multiColl' was built under R version 4.0.3
\end{verbatim}

\begin{Shaded}
\begin{Highlighting}[]
\CommentTok{# load the air quality data}
\KeywordTok{load}\NormalTok{(}\StringTok{"Data/Airq_numeric.Rdata"}\NormalTok{)}

\CommentTok{# set to dataframe}
\NormalTok{dfAirQ <-}\StringTok{ }\KeywordTok{data.frame}\NormalTok{(Airq)}

\CommentTok{# select dependent variable of air quality}
\NormalTok{Yair =}\StringTok{ }\NormalTok{dfAirQ}\OperatorTok{$}\NormalTok{airq}

\CommentTok{# select all other variables as independent variables}
\NormalTok{Xair =}\StringTok{ }\NormalTok{dfAirQ[,}\OperatorTok{-}\DecValTok{1}\NormalTok{]}

\CommentTok{# scale the independent variables, and add an intercept to these}
\NormalTok{XairScaled <-}\StringTok{ }\KeywordTok{scale}\NormalTok{(Xair)}
\NormalTok{XairIntercept <-}\StringTok{ }\KeywordTok{cbind}\NormalTok{(}\DataTypeTok{intercept =} \DecValTok{1}\NormalTok{, XairScaled)}

\CommentTok{# set the data to matrix format}
\NormalTok{mYair <-}\StringTok{ }\KeywordTok{as.matrix}\NormalTok{(Yair)}
\NormalTok{mXairIntercept <-}\StringTok{ }\KeywordTok{as.matrix}\NormalTok{(XairIntercept)}

\CommentTok{# set seed to ensure stability of results}
\KeywordTok{set.seed}\NormalTok{(}\DecValTok{0}\NormalTok{)}

\CommentTok{# set e small}
\NormalTok{e <-}\StringTok{ }\FloatTok{0.0000000001}


\CommentTok{# calculate the model with MM, for 1-5 variables. This contains all the values shown in the paper }
\NormalTok{compareModelMM <-}\StringTok{ }\KeywordTok{findModelMM}\NormalTok{(mXairIntercept, mYair, e)}
\end{Highlighting}
\end{Shaded}

\hypertarget{refs}{}
\leavevmode\hypertarget{ref-baklanov2016megacities}{}%
Baklanov, Alexander, Luisa T Molina, and Michael Gauss. 2016.
``Megacities, Air Quality and Climate.'' \emph{Atmospheric Environment}
126: 235--49.

\leavevmode\hypertarget{ref-leelHossy2014dispersion}{}%
Leelőssy, Ádám, Ferenc Molnár, Ferenc Izsák, Ágnes Havasi, István Lagzi,
and Róbert Mészáros. 2014. ``Dispersion Modeling of Air Pollutants in
the Atmosphere: A Review.'' \emph{Central European Journal of
Geosciences} 6 (3): 257--78.

\leavevmode\hypertarget{ref-liu2019nonlinear}{}%
Liu, Yuncai, Nengsheng Luo, and Shusheng Wu. 2019. ``Nonlinear Effects
of Environmental Regulation on Environmental Pollution.'' \emph{Discrete
Dynamics in Nature and Society} 2019.

\leavevmode\hypertarget{ref-RstudioData}{}%
r-project. 2020. ``Ecdat: Data Sets for Econometrics.'' World Health
Organization.
\url{https://cran.r-project.org/web/packages/Ecdat/index.html}.

\leavevmode\hypertarget{ref-WHOestimation}{}%
WHO. 2020. ``Air Pollution.'' World Health Organization.
\url{https://www.who.int/health-topics/air-pollution\#tab=tab_1}.

\leavevmode\hypertarget{ref-xiong2014better}{}%
Xiong, Shifeng. 2014. ``Better Subset Regression.'' \emph{Biometrika}
101 (1): 71--84.

\end{document}
